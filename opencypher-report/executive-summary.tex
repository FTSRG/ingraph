% !TeX spellcheck = en_GB
% !TeX encoding = UTF-8
\chapter*{Executive Summary}
\label{chp:executive-summary}

This document is generated on each commit of the \ingraph repository\footnote{\url{https://github.com/FTSRG/ingraph}}.

\paragraph{Structure.} \autoref{chp:foundations} introduces the theoretical foundations of the \opencypher language. \autoref{chp:incremental} presents incremental relational operators.

\paragraph{Appendices.} The appendix chapters contain sets of Cypher queries, and their representations as relational algebraic expressions and trees, along with their incremental equivalents.

\begin{itemize}
	\item \autoref{chp:tck}: the acceptance tests defined in the \opencypher Technology Compliance Kit\footnote{\url{https://github.com/opencypher/openCypher/tree/master/tck}}.
	\item \autoref{chp:fraud-detection}: fraud detection queries based on the Neo4j white paper.
	\item \autoref{chp:ldbc-snb-interactive}: LDBC Social Network Benchmark's Interactive queries.
  \item \autoref{chp:ldbc-snb-bi}: LDBC Social Network Benchmark's BI queries.
	\item \autoref{chp:movie-database}: Movie Database queries from the Neo4j tutorials.
	\item \autoref{chp:static-analysis-java}: Java static analysis queries.
	\item \autoref{chp:static-analysis-javascript}: JavaScript static analysis queries.
	\item \autoref{chp:trainbenchmark}: Train Benchmark queries.
	\item \autoref{chp:adbis-examples}: some examples for the ADBIS paper.
\end{itemize}
