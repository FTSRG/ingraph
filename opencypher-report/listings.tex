%if lstlistings is used
%better approach: use the minted package - see https://en.wikibooks.org/wiki/LaTeX/Source_Code_Listings#The_minted_package
%SzG: updating minted is quite cumbersome, see http://tex.stackexchange.com/questions/18083/how-to-add-custom-c-keywords-to-be-recognized-by-minted
\usepackage{listings}

\definecolor{lightgray}{RGB}{242,242,242}
\definecolor{keywordcolor}{RGB}{0,0,160}
\definecolor{commentcolor}{RGB}{0,128,64}
\definecolor{stringcolor}{RGB}{0,128,0}
\lstset{
	numbers=left,
	numberstyle=\scriptsize\ttfamily,
	stepnumber=1,
	numbersep=5pt,
	%
	backgroundcolor=\color{lightgray},
	basicstyle=\scriptsize\ttfamily, % print whole listing small
	keywordstyle=\color{keywordcolor}\bfseries\ttfamily,
	commentstyle=\color{commentcolor}\ttfamily,
	stringstyle=\color{stringcolor}\ttfamily,
	identifierstyle=, % nothing happens
	%
	showstringspaces=false, % no special string spaces
	aboveskip=3pt,
	belowskip=3pt,
	columns=flexible,
	keepspaces=true,
	breaklines=true,
	frameround=tttt,
	captionpos=b,
	tabsize=2,
	frame=tb,
	framerule=0pt,
	framexleftmargin=0.25em,
%	upquote=true
}

\lstdefinelanguage{cypher}
{
	morekeywords={
		MATCH, OPTIONAL, WHERE, NOT, AND, OR, XOR, RETURN, DISTINCT, ORDER, BY, ASC, ASCENDING, DESC, DESCENDING, UNWIND, AS, UNION, WITH, ALL, CREATE, DELETE, DETACH, REMOVE, SET, MERGE, SET, SKIP, LIMIT,
    % string operators
    STARTS, ENDS, CONTAINS,
    % set operator
    IN
		% some legacy rules
		INDEX, DROP, UNIQUE, CONSTRAINT, EXPLAIN, PROFILE, START, CASE,
		% some SQL-only keywords
		GROUP, HAVING,
	},
	sensitive=true,
	morecomment=[l]{//},
	morecomment=[s]{/*}{*/},
	morestring=[b]{"},
}

\newcommand{\listingcypher}[2]{
	\lstset{
		language=Cypher
	}
	\lstinputlisting[label=lst:#1, caption=#2.]{queries/#1.cyp}
}

\definecolor{relationship}{RGB}{171,0,85}
\lstset{
	basicstyle=\small\ttfamily,
	tabsize=4,
	morekeywords={Class, encapsulates, Feature},
%	literate=*
%	{[:gathers]}{\textcolor{relationship}{[:gathers]}}{1}
}


\lstset{language=Cypher,
	literate=*
	{<v>}{\guillemotleft{}}{1}
	{</v>}{\guillemotright{}}{1},
%	{<v>}{\guilsinglleft{}}{1}
%	{</v>}{\guilsinglright{}}{1},
}
