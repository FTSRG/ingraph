\chapter{Fraud Detection}
\label{chp:fraud-detection}

\section{Queries}

\subsection{create}

\subsubsection*{Query specification}

\begin{lstlisting}
// Create account holders
CREATE (accountHolder1:AccountHolder {
  FirstName: "John",
  LastName: "Doe",
  UniqueId: "JohnDoe" })

CREATE (accountHolder2:AccountHolder {
  FirstName: "Jane",
  LastName: "Appleseed",
  UniqueId: "JaneAppleseed" })

CREATE (accountHolder3:AccountHolder {
  FirstName: "Matt",
  LastName: "Smith",
  UniqueId: "MattSmith" })

// Create Address
CREATE (address1:Address {
  Street: "123 NW 1st Street",
  City: "San Francisco",
  State: "California",
  ZipCode: "94101" })

// Connect 3 account holders to 1 address
CREATE
  (accountHolder1)-[:HAS_ADDRESS]->(address1),
  (accountHolder2)-[:HAS_ADDRESS]->(address1),
  (accountHolder3)-[:HAS_ADDRESS]->(address1)

// Create Phone Number
CREATE (phoneNumber1:PhoneNumber { PhoneNumber: "555-555-5555" })

// Connect 2 account holders to 1 phone number
CREATE
  (accountHolder1)-[:HAS_PHONENUMBER]->(phoneNumber1),
  (accountHolder2)-[:HAS_PHONENUMBER]->(phoneNumber1)

// Create SSN
CREATE (ssn1:SSN { SSN: "241-23-1234" })

// Connect 2 account holders to 1 SSN
CREATE
  (accountHolder2)-[:HAS_SSN]->(ssn1),
  (accountHolder3)-[:HAS_SSN]->(ssn1)

// Create SSN and connect 1 account holder
CREATE (ssn2:SSN { SSN: "241-23-4567" })<-[:HAS_SSN]-(accountHolder1)

// Create Credit Card and connect 1 account holder
CREATE (creditCard1:CreditCard {
  AccountNumber: "1234567890123456",
  Limit: 5000, Balance: 1442.23,
  ExpirationDate: "01-20",
  SecurityCode: "123" })<-[:HAS_CREDITCARD]-(accountHolder1)

// Create Bank Account and connect 1 account holder
CREATE (bankAccount1:BankAccount {
  AccountNumber: "2345678901234567",
  Balance: 7054.43 })<-[:HAS_BANKACCOUNT]-(accountHolder1)

// Create Credit Card and connect 1 account holder
CREATE (creditCard2:CreditCard {
  AccountNumber: "1234567890123456",
  Limit: 4000, Balance: 2345.56,
  ExpirationDate: "02-20",
  SecurityCode: "456" })<-[:HAS_CREDITCARD]-(accountHolder2)

// Create Bank Account and connect 1 account holder
CREATE (bankAccount2:BankAccount {
  AccountNumber: "3456789012345678",
  Balance: 4231.12 })<-[:HAS_BANKACCOUNT]-(accountHolder2)

// Create Unsecured Loan and connect 1 account holder
CREATE (unsecuredLoan2:UnsecuredLoan {
  AccountNumber: "4567890123456789-0",
  Balance: 9045.53,
  APR: .0541,
  LoanAmount: 12000.00 })<-[:HAS_UNSECUREDLOAN]-(accountHolder2)

// Create Bank Account and connect 1 account holder
CREATE (bankAccount3:BankAccount {
  AccountNumber: "4567890123456789",
  Balance: 12345.45 })<-[:HAS_BANKACCOUNT]-(accountHolder3)

// Create Unsecured Loan and connect 1 account holder
CREATE (unsecuredLoan3:UnsecuredLoan {
  AccountNumber: "5678901234567890-0",
  Balance: 16341.95, APR: .0341,
  LoanAmount: 22000.00 })<-[:HAS_UNSECUREDLOAN]-(accountHolder3)

// Create Phone Number and connect 1 account holder
CREATE (phoneNumber2:PhoneNumber {
  PhoneNumber: "555-555-1234" })<-[:HAS_PHONENUMBER]-(accountHolder3)

RETURN *
\end{lstlisting}

\subsubsection*{Relational algebra expression}

\begin{align*}
\begin{autobreak}
\text{Cannot convert to expression.}
\end{autobreak}
\end{align*}

\subsubsection*{Relational algebra tree}

\adjustbox{max width=\textwidth}{%
\text{Cannot visualize tree.}
}

\subsubsection*{Relational algebra tree for incremental queries}

\adjustbox{max width=\textwidth}{%
Cannot visualize incremental tree.
}
\subsection{financial-risk}

\subsubsection*{Query specification}

\begin{lstlisting}
MATCH (accountHolder:AccountHolder)-[]->(contactInformation)
WITH
  contactInformation,
  count(accountHolder) AS RingSize

MATCH
  (contactInformation)<-[]-(accountHolder),
  (accountHolder)-[r:HAS_CREDITCARD|HAS_UNSECUREDLOAN]->(unsecuredAccount)
WITH
  collect(DISTINCT accountHolder.UniqueId) AS AccountHolders,
  contactInformation, RingSize,
  SUM(
    CASE type(r)
      WHEN 'HAS_CREDITCARD' THEN unsecuredAccount.LIMIT
      WHEN 'HAS_UNSECUREDLOAN' THEN unsecuredAccount.Balance
    ELSE 0
  END) AS FinancialRisk
WHERE RingSize > 1

RETURN
  AccountHolders AS FraudRing,
  labels(contactInformation) AS ContactType,
  RingSize,
  round(FinancialRisk) AS FinancialRisk

ORDER BY FinancialRisk DESC
\end{lstlisting}

\subsubsection*{Relational algebra expression}

\begin{align*}
\begin{autobreak}
\text{Cannot convert to expression.}
\end{autobreak}
\end{align*}

\subsubsection*{Relational algebra tree}

\adjustbox{max width=\textwidth}{%
\text{Cannot visualize tree.}
}

\subsubsection*{Relational algebra tree for incremental queries}

\adjustbox{max width=\textwidth}{%
Cannot visualize incremental tree.
}
\subsection{shared-contact-information}

\subsubsection*{Query specification}

\begin{lstlisting}
MATCH (accountHolder:AccountHolder)-[]->(contactInformation)
WITH
  contactInformation,
  count(accountHolder) AS RingSize

MATCH (contactInformation)<-[]-(accountHolder)
WITH
  collect(accountHolder.UniqueId) AS AccountHolders,
  contactInformation, RingSize
WHERE RingSize > 1

RETURN
  AccountHolders AS FraudRing,
  labels(contactInformation) AS ContactType,
  RingSize
ORDER BY
  RingSize DESC
\end{lstlisting}

\subsubsection*{Relational algebra expression}

\begin{align*}
\begin{autobreak}
\text{Cannot convert to expression.}
\end{autobreak}
\end{align*}

\subsubsection*{Relational algebra tree}

\adjustbox{max width=\textwidth}{%
\text{Cannot visualize tree.}
}

\subsubsection*{Relational algebra tree for incremental queries}

\adjustbox{max width=\textwidth}{%
Cannot visualize incremental tree.
}
