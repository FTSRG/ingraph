\chapter{LDBC Social Network Benchmark}
\label{chp:ldbc-snb}

\section{query-10}

\subsection*{Query specification}

\begin{lstlisting}
MATCH (person:Person)-[:KNOWS*2..2]-(friend:Person)-[:IS_LOCATED_IN]->(city:City)
WHERE ((friend.birthday_month = $month AND friend.birthday_day >= 21)
   OR (friend.birthday_month = ($month+1) % 12 AND friend.birthday_day < 22))
  AND NOT(friend = person) // I think this condition is unnecessary as Cypher will not travel the same edge twice (szarnyasg)
  AND NOT((friend)-[:KNOWS]-(person))
WITH DISTINCT friend, city, person
OPTIONAL MATCH (friend)<-[:HAS_CREATOR]-(post:Post)
WITH
  friend,
  city,
  collect(post) AS posts,
  person
WITH
  friend,
  city,
  length(posts) AS postCount,
  length([p IN posts WHERE (p)-[:HAS_TAG]->(:Tag)<-[:HAS_INTEREST]-(person)]) AS commonPostCount
RETURN
  friend.id AS personId,
  friend.firstName AS personFirstName,
  friend.lastName AS personLastName,
  friend.gender AS personGender,
  city.name AS personCityName,
  commonPostCount - (postCount - commonPostCount) AS commonInterestScore
ORDER BY
  commonInterestScore DESC,
  personId ASC
LIMIT 10
\end{lstlisting}

\subsection*{Relational algebra expression}

\begin{flalign*}
\text{Cannot convert to expression.}
\end{flalign*}

\subsection*{Relational algebra tree}
\adjustbox{max width=\textwidth}{%
\text{Cannot visualize tree.}
}

\subsection*{Relational algebra tree for incremental queries}
\adjustbox{max width=\textwidth}{%
Cannot visualize incremental tree.
}

\section{query-11}

\subsection*{Query specification}

\begin{lstlisting}
MATCH (person:Person)-[:KNOWS*1..2]-(friend:Person)
WHERE NOT(person = friend) // I think this condition is unnecessary as Cypher will not travel the same edge twice (szarnyasg)
WITH DISTINCT friend
MATCH (friend)-[worksAt:WORKS_AT]->(company:Company)-[:IS_LOCATED_IN]->(:Country)
WHERE worksAt.workFrom < $date
RETURN
  friend.id AS friendId,
  friend.firstName AS friendFirstName,
  friend.lastName AS friendLastName,
  worksAt.workFrom AS workFromYear,
  company.name AS companyName
ORDER BY
  workFromYear ASC,
  friendId ASC,
  companyName DESC
LIMIT 10
\end{lstlisting}

\subsection*{Relational algebra expression}

\begin{flalign*}
\text{Cannot convert to expression.}
\end{flalign*}

\subsection*{Relational algebra tree}
\adjustbox{max width=\textwidth}{%
\text{Cannot visualize tree.}
}

\subsection*{Relational algebra tree for incremental queries}
\adjustbox{max width=\textwidth}{%
Cannot visualize incremental tree.
}

\section{query-12}

\subsection*{Query specification}

\begin{lstlisting}
MATCH (:Person)-[:KNOWS]-(friend:Person)
OPTIONAL MATCH (friend)<-[:HAS_CREATOR]-(comment:Comment)-[:REPLY_OF]->(:Post)-[:HAS_TAG]->(tag:Tag),
(tag)-[:HAS_TYPE]->(tagClass:TagClass)-[:IS_SUBCLASS_OF*0..]->(baseTagClass:TagClass)
WHERE tagClass.name = $class
   OR baseTagClass.name = $class
RETURN
  friend.id AS friendId,
  friend.firstName AS friendFirstName,
  friend.lastName AS friendLastName,
  collect(DISTINCT tag.name) AS tagNames,
  count(DISTINCT comment) AS count
ORDER BY
  count DESC,
  friendId ASC
LIMIT 10
\end{lstlisting}

\subsection*{Relational algebra expression}

\begin{flalign*}
\text{Cannot convert to expression.}
\end{flalign*}

\subsection*{Relational algebra tree}
\adjustbox{max width=\textwidth}{%
\text{Cannot visualize tree.}
}

\subsection*{Relational algebra tree for incremental queries}
\adjustbox{max width=\textwidth}{%
Cannot visualize incremental tree.
}

\section{query-3}

\subsection*{Query specification}

\begin{lstlisting}
MATCH
  (person:Person)-[:KNOWS*1..2]-(friend:Person)<-[:HAS_CREATOR]-(messageX),
  (messageX)-[:IS_LOCATED_IN]->(countryX:Country)
WHERE NOT(person = friend) // I think this condition is unnecessary as Cypher will not travel the same edge twice (szarnyasg)
  AND NOT((friend)-[:IS_LOCATED_IN]->()-[:IS_PART_OF]->(countryX))
  AND countryX.name = $countryXName
  AND messageX.creationDate >= $date1
  AND messageX.creationDate < $date2
WITH friend, count(DISTINCT messageX) AS xCount
MATCH (friend)<-[:HAS_CREATOR]-(messageY)-[:IS_LOCATED_IN]->(countryY:Country)
WHERE countryY.name = $countryYName
  AND NOT((friend)-[:IS_LOCATED_IN]->()-[:IS_PART_OF]->(countryY))
  AND messageY.creationDate >= $date1
  AND messageY.creationDate < $date2
WITH
  friend.id AS friendId,
  friend.firstName AS friendFirstName,
  friend.lastName AS friendLastName,
  xCount,
  count(DISTINCT messageY) AS yCount
RETURN
  friendId,
  friendFirstName,
  friendLastName,
  xCount,
  yCount,
  xCount + yCount AS xyCount
ORDER BY
  xyCount DESC,
  friendId ASC
LIMIT 10
\end{lstlisting}

\subsection*{Relational algebra expression}

\begin{flalign*}
\text{Cannot convert to expression.}
\end{flalign*}

\subsection*{Relational algebra tree}
\adjustbox{max width=\textwidth}{%
\text{Cannot visualize tree.}
}

\subsection*{Relational algebra tree for incremental queries}
\adjustbox{max width=\textwidth}{%
Cannot visualize incremental tree.
}

\section{query-4}

\subsection*{Query specification}

\begin{lstlisting}
MATCH (person:Person)-[:KNOWS]-(:Person)<-[:HAS_CREATOR]-(post:Post)-[HAS_TAG]->(tag:Tag)
WHERE post.creationDate >= $date1
  AND post.creationDate < $date2
OPTIONAL MATCH (tag)<-[:HAS_TAG]-(oldPost:Post)
WHERE oldPost.creationDate < $date1
WITH tag, post, length(collect(oldPost)) AS oldPostCount
WHERE oldPostCount=0
RETURN
  tag.name AS tagName,
  length(collect(post)) AS postCount
ORDER BY
  postCount DESC,
  tagName ASC
LIMIT 10
\end{lstlisting}

\subsection*{Relational algebra expression}

\begin{flalign*}
\text{Cannot convert to expression.}
\end{flalign*}

\subsection*{Relational algebra tree}
\adjustbox{max width=\textwidth}{%
\text{Cannot visualize tree.}
}

\subsection*{Relational algebra tree for incremental queries}
\adjustbox{max width=\textwidth}{%
Cannot visualize incremental tree.
}

\section{query-5}

\subsection*{Query specification}

\begin{lstlisting}
MATCH (person:Person)-[:KNOWS*1..2]-(friend:Person)<-[membership:HAS_MEMBER]-(forum:Forum)
WHERE membership.joinDate > $date
  AND NOT(person = friend) // I think this condition is unnecessary as Cypher will not travel the same edge twice (szarnyasg)
WITH DISTINCT friend, forum
OPTIONAL MATCH (friend)<-[:HAS_CREATOR]-(post:Post)<-[:CONTAINER_OF]-(forum)
WITH forum, count(post) AS postCount
RETURN
  forum.title AS forumName,
  postCount
ORDER BY
  postCount DESC,
  forum.id ASC
LIMIT 10
\end{lstlisting}

\subsection*{Relational algebra expression}

\begin{flalign*}
\text{Cannot convert to expression.}
\end{flalign*}

\subsection*{Relational algebra tree}
\adjustbox{max width=\textwidth}{%
\text{Cannot visualize tree.}
}

\subsection*{Relational algebra tree for incremental queries}
\adjustbox{max width=\textwidth}{%
Cannot visualize incremental tree.
}

\section{query-6}

\subsection*{Query specification}

\begin{lstlisting}
MATCH
  (person:Person)-[:KNOWS*1..2]-(friend:Person),
  (friend)<-[:HAS_CREATOR]-(friendPost:Post)-[:HAS_TAG]->(knownTag:Tag)
WHERE NOT(person = friend) // I think this condition is unnecessary as Cypher will not travel the same edge twice (szarnyasg)
MATCH (friendPost)-[:HAS_TAG]->(commonTag:Tag)
WHERE NOT(commonTag = knownTag)
WITH DISTINCT commonTag, knownTag, friend
MATCH (commonTag)<-[:HAS_TAG]-(commonPost:Post)-[:HAS_TAG]->(knownTag)
WHERE (commonPost)-[:HAS_CREATOR]->(friend)
RETURN
  commonTag.name AS tagName,
  count(commonPost) AS postCount
ORDER BY
  postCount DESC,
  tagName ASC
LIMIT 10
\end{lstlisting}

\subsection*{Relational algebra expression}

\begin{flalign*}
\text{Cannot convert to expression.}
\end{flalign*}

\subsection*{Relational algebra tree}
\adjustbox{max width=\textwidth}{%
\text{Cannot visualize tree.}
}

\subsection*{Relational algebra tree for incremental queries}
\adjustbox{max width=\textwidth}{%
Cannot visualize incremental tree.
}

\section{query-7}

\subsection*{Query specification}

\begin{lstlisting}
MATCH (person:Person)<-[:HAS_CREATOR]-(message)<-[like:LIKES]-(liker:Person)
WITH liker, message, like.creationDate AS likeTime, person
ORDER BY
  likeTime DESC,
  message.id ASC
WITH liker, head(collect({msg: message, likeTime: likeTime})) AS latestLike, person
RETURN
  liker.id AS personId,
  liker.firstName AS personFirstName,
  liker.lastName AS personLastName,
  latestLike.likeTime AS likeTime,
  NOT((liker)-[:KNOWS]-(person)) AS isNew,
  latestLike.msg.id AS messageId,
  latestLike.msg.content AS messageContent,
  latestLike.likeTime - latestLike.msg.creationDate AS latencyAsMilli
ORDER BY
  likeTime DESC,
  personId ASC
LIMIT 10
\end{lstlisting}

\subsection*{Relational algebra expression}

\begin{flalign*}
\text{Cannot convert to expression.}
\end{flalign*}

\subsection*{Relational algebra tree}
\adjustbox{max width=\textwidth}{%
\text{Cannot visualize tree.}
}

\subsection*{Relational algebra tree for incremental queries}
\adjustbox{max width=\textwidth}{%
Cannot visualize incremental tree.
}

\section{query-8}

\subsection*{Query specification}

\begin{lstlisting}
MATCH (start:Person)<-[:HAS_CREATOR]-()<-[:REPLY_OF]-(comment:Comment)-[:HAS_CREATOR]->(person:Person)
RETURN
  person.id AS personId,
  person.firstName AS personFirstName,
  person.lastName AS personLastName,
  comment.id AS commentId,
  comment.creationDate AS commentCreationDate,
  comment.content AS commentContent
ORDER BY
  commentCreationDate DESC,
  commentId ASC
LIMIT 10
\end{lstlisting}

\subsection*{Relational algebra expression}

\begin{flalign*}
\projection{\var{person.id},~\var{person.firstName},~\var{person.lastName},~\var{comment.id},~\var{comment.creationDate},~\var{comment.content}} \Big(\alldifferent{} \Big(\expandout{comment}{person}{Person}{\_e823}{HAS\_CREATOR}{}{} \Big(\expandin{\_e820}{comment}{Comment}{\_e822}{REPLY\_OF}{}{} \Big(\expandin{start}{\_e820}{}{\_e821}{HAS\_CREATOR}{}{} \Big(\getvertices{start}{Person}\Big)\Big)\Big)\Big)\Big)
\end{flalign*}

\subsection*{Relational algebra tree}
\adjustbox{max width=\textwidth}{%
\text{Cannot visualize tree.}
}

\subsection*{Relational algebra tree for incremental queries}
\adjustbox{max width=\textwidth}{%
\begin{forest} for tree={align=center}
[
	{$\projection{\var{person.id},~\var{person.firstName},~\var{person.lastName},~\var{comment.id},~\var{comment.creationDate},~\var{comment.content}}$
			\\
			\footnotesize
			$\color{gray} \langle \var{} \rangle$
			}
[
	{$\alldifferent{}$
			\\
			\footnotesize
			$\color{gray} \langle \var{} \rangle$
			}
[
	{$\expandout{comment}{person}{Person}{\_e831}{HAS\_CREATOR}{}{}$
			\\
			\footnotesize
			$\color{gray} \langle \var{} \rangle$
			}
[
	{$\expandin{\_e828}{comment}{Comment}{\_e830}{REPLY\_OF}{}{}$
			\\
			\footnotesize
			$\color{gray} \langle \var{} \rangle$
			}
[
	{$\expandin{start}{\_e828}{}{\_e829}{HAS\_CREATOR}{}{}$
			\\
			\footnotesize
			$\color{gray} \langle \var{} \rangle$
			}
[
	{$\getvertices{start}{Person}$
			\\
			\footnotesize
			$\color{gray} \langle \var{} \rangle$
			},tier=input,for tree={blue,densely dashed}
]
]
]
]
]
]
;
\end{forest}
}

