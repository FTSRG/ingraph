\chapter{LDBC Social Network Benchmark}
\label{chp:ldbc-snb}

\section{query-3}

\subsection*{Query specification}

\begin{lstlisting}
// Q3. Friends within 2 steps that recently traveled to countries X and Y. Find top 20 friends and friends of friends of a given Person who have made a post or a comment in the foreign CountryX and CountryY within a specified period of DurationInDays after a startDate. Sorted results descending by total number of posts.
MATCH
  (person:Person)-[:KNOWS*1..2]-(friend:Person)<-[:HAS_CREATOR]-(messageX),
  (messageX)-[:IS_LOCATED_IN]->(countryX:Country)
WHERE NOT(person = friend) // I think this condition is unnecessary as Cypher will not travel the same edge twice (szarnyasg)
  AND NOT((friend)-[:IS_LOCATED_IN]->()-[:IS_PART_OF]->(countryX))
  AND countryX.name = $countryXName
  AND messageX.creationDate >= $date1
  AND messageX.creationDate < $date2
WITH friend, count(DISTINCT messageX) AS xCount
MATCH (friend)<-[:HAS_CREATOR]-(messageY)-[:IS_LOCATED_IN]->(countryY:Country)
WHERE countryY.name = $countryYName
  AND NOT((friend)-[:IS_LOCATED_IN]->()-[:IS_PART_OF]->(countryY))
  AND messageY.creationDate >= $date1
  AND messageY.creationDate < $date2
WITH
  friend.id AS friendId,
  friend.firstName AS friendFirstName,
  friend.lastName AS friendLastName,
  xCount,
  count(DISTINCT messageY) AS yCount
RETURN
  friendId,
  friendFirstName,
  friendLastName,
  xCount,
  yCount,
  xCount + yCount AS xyCount
ORDER BY
  xyCount DESC,
  friendId ASC
LIMIT 10
\end{lstlisting}

\subsection*{Relational algebra expression}

\begin{flalign*}
\text{Cannot convert to expression.}
\end{flalign*}

\subsection*{Relational algebra tree}
\adjustbox{max width=\textwidth}{%
\text{Cannot visualize tree.}
}

\subsection*{Relational algebra tree for incremental queries}
\adjustbox{max width=\textwidth}{%
Cannot visualize incremental tree.
}

\section{query-4}

\subsection*{Query specification}

\begin{lstlisting}
// Q4. New Topics. Given a start Person, find the top 10 most popular Tags (by total number of posts with the tag) that are attached to Posts that were created by that Person's friends within a given time interval.
MATCH (person:Person)-[:KNOWS]-(:Person)<-[:HAS_CREATOR]-(post:Post)-[HAS_TAG]->(tag:Tag)
WHERE post.creationDate >= $date1
  AND post.creationDate < $date2
OPTIONAL MATCH (tag)<-[:HAS_TAG]-(oldPost:Post)
WHERE oldPost.creationDate < $date1
WITH tag, post, length(collect(oldPost)) AS oldPostCount
WHERE oldPostCount=0
RETURN
  tag.name AS tagName,
  length(collect(post)) AS postCount
ORDER BY
  postCount DESC,
  tagName ASC
LIMIT 10
\end{lstlisting}

\subsection*{Relational algebra expression}

\begin{flalign*}
\text{Cannot convert to expression.}
\end{flalign*}

\subsection*{Relational algebra tree}
\adjustbox{max width=\textwidth}{%
\text{Cannot visualize tree.}
}

\subsection*{Relational algebra tree for incremental queries}
\adjustbox{max width=\textwidth}{%
Cannot visualize incremental tree.
}

\section{query-5}

\subsection*{Query specification}

\begin{lstlisting}
// Q5. New groups. Given a start Person, find the top 20 Forums the friends and friends of friends of that Person joined after a given Date. Sort results descending by the number of Posts in each Forum that were created by any of these Persons.
MATCH (person:Person)-[:KNOWS*1..2]-(friend:Person)<-[membership:HAS_MEMBER]-(forum:Forum)
WHERE membership.joinDate > $date
  AND NOT(person = friend) // I think this condition is unnecessary as Cypher will not travel the same edge twice (szarnyasg)
WITH DISTINCT friend, forum
OPTIONAL MATCH (friend)<-[:HAS_CREATOR]-(post:Post)<-[:CONTAINER_OF]-(forum)
WITH forum, count(post) AS postCount
RETURN
  forum.title AS forumName,
  postCount
ORDER BY
  postCount DESC,
  forum.id ASC
LIMIT 10
\end{lstlisting}

\subsection*{Relational algebra expression}

\begin{flalign*}
\text{Cannot convert to expression.}
\end{flalign*}

\subsection*{Relational algebra tree}
\adjustbox{max width=\textwidth}{%
\text{Cannot visualize tree.}
}

\subsection*{Relational algebra tree for incremental queries}
\adjustbox{max width=\textwidth}{%
Cannot visualize incremental tree.
}

\section{query-6}

\subsection*{Query specification}

\begin{lstlisting}
// Q6. Tag co-occurrence. Given a start Person and some Tag, find the other Tags that occur together with this Tag on Posts that were created by Person's friends and friends of friends. Return top 10 Tags, sorted descending by the count of Posts that were created by these Persons, which contain both this Tag and the given Tag.
MATCH
  (person:Person)-[:KNOWS*1..2]-(friend:Person),
  (friend)<-[:HAS_CREATOR]-(friendPost:Post)-[:HAS_TAG]->(knownTag:Tag)
WHERE NOT(person = friend) // I think this condition is unnecessary as Cypher will not travel the same edge twice (szarnyasg)
MATCH (friendPost)-[:HAS_TAG]->(commonTag:Tag)
WHERE NOT(commonTag = knownTag)
WITH DISTINCT commonTag, knownTag, friend
MATCH (commonTag)<-[:HAS_TAG]-(commonPost:Post)-[:HAS_TAG]->(knownTag)
WHERE (commonPost)-[:HAS_CREATOR]->(friend)
RETURN
  commonTag.name AS tagName,
  count(commonPost) AS postCount
ORDER BY
  postCount DESC,
  tagName ASC
LIMIT 10
\end{lstlisting}

\subsection*{Relational algebra expression}

\begin{flalign*}
\text{Cannot convert to expression.}
\end{flalign*}

\subsection*{Relational algebra tree}
\adjustbox{max width=\textwidth}{%
\text{Cannot visualize tree.}
}

\subsection*{Relational algebra tree for incremental queries}
\adjustbox{max width=\textwidth}{%
Cannot visualize incremental tree.
}

\section{query-7}

\subsection*{Query specification}

\begin{lstlisting}
// Q7. Recent likes. For the specified Person get the most recent likes of any of the person's posts, and the latency between the corresponding post and the like. Flag Likes from outside the direct connections. Return top 20 Likes, ordered descending by creation date of the like.
MATCH (person:Person)<-[:HAS_CREATOR]-(message)<-[like:LIKES]-(liker:Person)
WITH liker, message, like.creationDate AS likeTime, person
ORDER BY
  likeTime DESC,
  message.id ASC
WITH liker, head(collect({msg: message, likeTime: likeTime})) AS latestLike, person
RETURN
  liker.id AS personId,
  liker.firstName AS personFirstName,
  liker.lastName AS personLastName,
  latestLike.likeTime AS likeTime,
  NOT((liker)-[:KNOWS]-(person)) AS isNew,
  latestLike.msg.id AS messageId,
  latestLike.msg.content AS messageContent,
  latestLike.likeTime - latestLike.msg.creationDate AS latencyAsMilli
ORDER BY
  likeTime DESC,
  personId ASC
LIMIT 10
\end{lstlisting}

\subsection*{Relational algebra expression}

\begin{flalign*}
\text{Cannot convert to expression.}
\end{flalign*}

\subsection*{Relational algebra tree}
\adjustbox{max width=\textwidth}{%
\text{Cannot visualize tree.}
}

\subsection*{Relational algebra tree for incremental queries}
\adjustbox{max width=\textwidth}{%
Cannot visualize incremental tree.
}

\section{query-8}

\subsection*{Query specification}

\begin{lstlisting}
// Q8. Most recent replies. This query retrieves the 20 most recent reply comments to all the posts and comments of Person, ordered descending by creation date.
MATCH (start:Person)<-[:HAS_CREATOR]-()<-[:REPLY_OF]-(comment:Comment)-[:HAS_CREATOR]->(person:Person)
RETURN
  person.id AS personId,
  person.firstName AS personFirstName,
  person.lastName AS personLastName,
  comment.id AS commentId,
  comment.creationDate AS commentCreationDate,
  comment.content AS commentContent
ORDER BY
  commentCreationDate DESC,
  commentId ASC
LIMIT 10
\end{lstlisting}

\subsection*{Relational algebra expression}

\begin{flalign*}
\text{Cannot convert to expression.}
\end{flalign*}

\subsection*{Relational algebra tree}
\adjustbox{max width=\textwidth}{%
\text{Cannot visualize tree.}
}

\subsection*{Relational algebra tree for incremental queries}
\adjustbox{max width=\textwidth}{%
Cannot visualize incremental tree.
}

\section{query-10}

\subsection*{Query specification}

\begin{lstlisting}
// Q10. Friend recommendation. Find top 10 friends of a friend who posts much about the interests of Person and little about not interesting topics for the user. The search is restricted by the candidate's horoscopeSign. Returns friends for whom the difference between the total number of their posts about the interests of the specified user and the total number of their posts about topics that are not interests of the user, is as large as possible. Sort the result descending by this difference.
MATCH (person:Person)-[:KNOWS*2..2]-(friend:Person)-[:IS_LOCATED_IN]->(city:City)
WHERE ((friend.birthday_month = $month AND friend.birthday_day >= 21)
   OR (friend.birthday_month = ($month+1) % 12 AND friend.birthday_day < 22))
  AND NOT(friend = person) // I think this condition is unnecessary as Cypher will not travel the same edge twice (szarnyasg)
  AND NOT((friend)-[:KNOWS]-(person))
WITH DISTINCT friend, city, person
OPTIONAL MATCH (friend)<-[:HAS_CREATOR]-(post:Post)
WITH
  friend,
  city,
  collect(post) AS posts,
  person
WITH
  friend,
  city,
  length(posts) AS postCount,
  length([p IN posts WHERE (p)-[:HAS_TAG]->(:Tag)<-[:HAS_INTEREST]-(person)]) AS commonPostCount
RETURN
  friend.id AS personId,
  friend.firstName AS personFirstName,
  friend.lastName AS personLastName,
  friend.gender AS personGender,
  city.name AS personCityName,
  commonPostCount - (postCount - commonPostCount) AS commonInterestScore
ORDER BY
  commonInterestScore DESC,
  personId ASC
LIMIT 10
\end{lstlisting}

\subsection*{Relational algebra expression}

\begin{flalign*}
\text{Cannot convert to expression.}
\end{flalign*}

\subsection*{Relational algebra tree}
\adjustbox{max width=\textwidth}{%
\text{Cannot visualize tree.}
}

\subsection*{Relational algebra tree for incremental queries}
\adjustbox{max width=\textwidth}{%
Cannot visualize incremental tree.
}

\section{query-11}

\subsection*{Query specification}

\begin{lstlisting}
// Q11. Job referral. Find top 10 friends of the specified Person, or a friend of her friend (excluding the specified person), who has long worked in a company in a specified Country. Sort ascending by start date, and then ascending by person identifier.
MATCH (person:Person)-[:KNOWS*1..2]-(friend:Person)
WHERE NOT(person = friend) // I think this condition is unnecessary as Cypher will not travel the same edge twice (szarnyasg)
WITH DISTINCT friend
MATCH (friend)-[worksAt:WORKS_AT]->(company:Company)-[:IS_LOCATED_IN]->(:Country)
WHERE worksAt.workFrom < $date
RETURN
  friend.id AS friendId,
  friend.firstName AS friendFirstName,
  friend.lastName AS friendLastName,
  worksAt.workFrom AS workFromYear,
  company.name AS companyName
ORDER BY
  workFromYear ASC,
  friendId ASC,
  companyName DESC
LIMIT 10
\end{lstlisting}

\subsection*{Relational algebra expression}

\begin{flalign*}
\text{Cannot convert to expression.}
\end{flalign*}

\subsection*{Relational algebra tree}
\adjustbox{max width=\textwidth}{%
\text{Cannot visualize tree.}
}

\subsection*{Relational algebra tree for incremental queries}
\adjustbox{max width=\textwidth}{%
Cannot visualize incremental tree.
}

\section{query-12}

\subsection*{Query specification}

\begin{lstlisting}
// Q12. Expert Search. Find friends of a Person who have replied the most to posts with a tag in a given TagCategory. Return top 20 persons, sorted descending by number of replies.
MATCH (:Person)-[:KNOWS]-(friend:Person)
OPTIONAL MATCH (friend)<-[:HAS_CREATOR]-(comment:Comment)-[:REPLY_OF]->(:Post)-[:HAS_TAG]->(tag:Tag),
(tag)-[:HAS_TYPE]->(tagClass:TagClass)-[:IS_SUBCLASS_OF*0..]->(baseTagClass:TagClass)
WHERE tagClass.name = $class
   OR baseTagClass.name = $class
RETURN
  friend.id AS friendId,
  friend.firstName AS friendFirstName,
  friend.lastName AS friendLastName,
  collect(DISTINCT tag.name) AS tagNames,
  count(DISTINCT comment) AS count
ORDER BY
  count DESC,
  friendId ASC
LIMIT 10
\end{lstlisting}

\subsection*{Relational algebra expression}

\begin{flalign*}
\text{Cannot convert to expression.}
\end{flalign*}

\subsection*{Relational algebra tree}
\adjustbox{max width=\textwidth}{%
\text{Cannot visualize tree.}
}

\subsection*{Relational algebra tree for incremental queries}
\adjustbox{max width=\textwidth}{%
Cannot visualize incremental tree.
}

