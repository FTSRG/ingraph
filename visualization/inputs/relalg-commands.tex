\newcommand{\lxor}{\oplus}

\newcommand{\asc}{\twoheaduparrow}
\newcommand{\desc}{\twoheaddownarrow}

\newcommand{\tuple}[1]{\langle #1 \rangle}
\newcommand{\concatenation}{\Vert}

\newcommand{\literal}[1]{\mathsf{#1}}
\newcommand{\atom}[1]{\mathsf{#1}}

\newcommand{\var}[1]{\mathtt{#1}}
\newcommand{\edgevariable}[2]{\left[\var{#1}
	\ifstrempty{#2}{}{\colon{\atom{#2}}}
	\right]}
\newcommand{\nodevariable}[2]{(\var{#1}
	\ifstrempty{#2}{}{\colon{\atom{#2}}}
	)}

% see http://tug.ctan.org/info/symbols/comprehensive/symbols-a4.pdf

% nullary operators

\newcommand{\getverticesop}{\bigcirc}
\newcommand{\getvertices}[2]{\getverticesop_{\nodevariable{#1}{#2}}}

%\newcommand{\getedgesop}{\mathbin{\text{\rotatebox[origin=c]{90}{$\multimapboth$}}}}
\newcommand{\getedgesop}{\Uparrow}
\newcommand{\getedges}[6]{\getedgesop_{\nodevariable{#1}{#2}}^{\nodevariable{#3}{#4}} \edgevariable{#5}{#6}}
\newcommand{\getedgesi}[4]{\getedgesop_{\nodevariable{#1}{#2}}^{\nodevariable{#3}{#4}}}
\newcommand{\getedgesii}[2]{\edgevariable{#1}{#2}}

% unary operators

\newcommand{\expandbody}[3]{~_{\nodevariable{#1}{}}^{\nodevariable{#2}{#3}}}

\newcommand{\expandbothop}{\updownarrow}
\newcommand{\expandboth}[5]{\expandbothop \expandbody{#1}{#2}{#3} \edgevariable{#4}{#5} }

\newcommand{\expandoutop}{\uparrow}
\newcommand{\expandout}[5]{\expandoutop \expandbody{#1}{#2}{#3} \edgevariable{#4}{#5} }

\newcommand{\expandinop}{\downarrow}
\newcommand{\expandin}[5]{\expandinop \expandbody{#1}{#2}{#3} \edgevariable{#4}{#5}}

\newcommand{\alldifferentop}{\not\fallingdotseq}
\newcommand{\alldifferent}[1]{\alldifferentop_{\atom{#1}}}

\newcommand{\duplicateeliminationop}{\delta}
\newcommand{\duplicateelimination}{\duplicateeliminationop}

\newcommand{\sortop}{\tau}

\newcommand{\sortarbitrary}[3]{\sortop^{#1}_{#2}{#3}}

\newcommand{\sort}[2]{\sortop_{#1}{#2}}
\newcommand{\sortasc}[2]{\sortopasc_{#1}{#2}}
\newcommand{\sortdesc}[2]{\sortopdesc_{#1}{#2}}

\newcommand{\projectionop}{\pi}
\newcommand{\projection}[1]{\projectionop_\atom{#1}}

\newcommand{\selectionop}{\sigma}
\newcommand{\selection}[1]{\selectionop_\atom{#1}}

\newcommand{\renameop}{\rho}
\newcommand{\rename}[1]{\renameop_\atom{#1}}

\newcommand{\groupingop}{\gamma}
\newcommand{\grouping}[1]{\groupingop_\atom{#1}}

\newcommand{\topop}{\lambda}
\newcommand{\topp}[2]{\topop_{#1}^{#2}}
\newcommand{\skipp}[1]{\topop^{#1}}
\newcommand{\limit}[1]{\topop_{#1}}

% see A Formal Presentation of MongoDB (Extended Version)
% by Elena Botoeva, Diego Calvanese, Benjamin Cogrel, Martin Rezk, Guohui Xiao
% https://arxiv.org/abs/1603.09291
\newcommand{\unwindop}{\omega}
\newcommand{\unwind}[1]{\unwindop #1}

% binary operators

\newcommand{\joinop}{\bowtie}
\newcommand{\join}{\joinop}

\newcommand{\leftouterjoinop}{\leftouterjoin}
\newcommand{\myleftouterjoin}{\leftouterjoinop}

\newcommand{\antijoinop}{\, \triangleright \,}
\newcommand{\antijoin}{\antijoinop}

\newcommand{\unionop}{\cup}
\newcommand{\union}{\unionop}

\newcommand{\bagunionop}{\uplus}
\newcommand{\bagunion}{\bagunionop}

\newcommand{\minusop}{\setminus}
\newcommand{\minus}{\minusop}

\newcommand{\intersectionop}{\cap}
\newcommand{\intersection}{\intersectionop}

\newcommand{\cartesianproductop}{\times}
\newcommand{\cartesianproduct}{\cartesianproduct}
